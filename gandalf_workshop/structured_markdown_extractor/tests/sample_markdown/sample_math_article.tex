\documentclass{article}
\usepackage{amsmath, amssymb} % For \mathbb

\title{On the Properties of Foo Bars}
\author{A. N. Other}
\date{2024-07-28}

\newtheorem{theorem}{Theorem}[section]
\newtheorem{definition}[theorem]{Definition}
\newtheorem{remark}[theorem]{Remark}

\begin{document}
\maketitle

\section{Introduction}

This paper introduces the concept of Foo Bars, a novel mathematical structure. We explore their basic properties and lay the groundwork for future research. The Foo Bar, denoted $\mathcal{F}_b$, is expected to unify several disparate concepts in modern algebra. Our investigation is primarily concerned with Foo Bars over a field $\mathbb{K}$.

\section{Definitions}

We begin with the formal definition of a Foo Bar.

\begin{definition}[Foo Bar]
\label{def:foobar}
A \textit{Foo Bar} over a field $\mathbb{K}$ is a tuple $(\mathcal{S}, \oplus, \otimes, \lambda_0)$ where:
\begin{enumerate}
    \item $\mathcal{S}$ is a non-empty set.
    \item $\oplus: \mathcal{S} \times \mathcal{S} \to \mathcal{S}$ is a binary operation called 'foo-addition'.
    \item $\otimes: \mathbb{K} \times \mathcal{S} \to \mathcal{S}$ is an operation called 'bar-multiplication'.
    \item $\lambda_0 \in \mathcal{S}$ is a distinguished element known as the 'base foo'.
\end{enumerate}
These components must satisfy Axioms F1-F3 (detailed elsewhere). For example, foo-addition $\oplus$ must be associative.
\end{definition}

\begin{remark}
\label{rem:fieldchoice}
The choice of the field $\mathbb{K}$ can significantly alter the properties of the Foo Bar. For instance, if $\mathbb{K} = \mathbb{R}$ (real numbers), the Foo Bar exhibits continuous properties, whereas if $\mathbb{K} = \mathbb{F}_p$ (finite field), it has discrete characteristics.
\end{remark}

\section{Main Theorem}

We now present the central theorem of this paper.

\begin{theorem}[Existence of Unique Baz Foo]
\label{thm:maintheorem}
Every Foo Bar $(\mathcal{S}, \oplus, \otimes, \lambda_0)$ over an algebraically closed field $\mathbb{K}$ of characteristic zero contains a unique non-trivial 'Baz Foo', denoted $\beta^*$, such that $\phi(\beta^*) = \lambda_0$ for a specific canonical map $\phi$.
\end{theorem}

\begin{proof}[Proof Sketch]
The proof proceeds in two parts:
\begin{enumerate}
    \item \textbf{Existence}: We construct an element $\beta_c$ using a sequence of foo-operations derived from the Zorn's Lemma applied to a partially ordered set of proto-Baz Foos. We then show $\beta_c$ satisfies the conditions for a Baz Foo.
    \item \textbf{Uniqueness}: Assume $\beta_1^*$ and $\beta_2^*$ are two distinct non-trivial Baz Foos. We apply the Bar Homomorphism Lemma (Lemma 3.0, not shown here) to demonstrate that $\beta_1^* = \beta_2^*$, leading to a contradiction.
\end{enumerate}
The full proof requires developing the theory of Bar Homomorphisms and is deferred to Appendix A.
\end{proof}

\section{Conclusion}

The introduction of Foo Bars and the proof of Theorem 3.1 open new avenues for research. Future work will focus on classifying Foo Bars over finite fields.

\end{document}
